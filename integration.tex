\documentclass{article}
\usepackage{amssymb}
\usepackage{amsmath}
\usepackage{amsthm}
\usepackage{mathtools}
\usepackage{hyperref}
\DeclarePairedDelimiter\abs{\lvert}{\rvert}
\newtheorem{proposition}{Proposition}
\def\E{\mathbb{E}}
\def\R{\mathbb{R}}
\DeclareMathOperator*{\tr}{tr}
\title{High order moment of uniform distribution on the sphere}
\author{Feng Zhao}
\begin{document}
\maketitle
Consider $X_1, \dots, X_n \sim N(0, 1)$.
Let $X = \sqrt{X_1^2 + \dots + X_n^2}$.
$Y_i = \frac{X_i}{X}$.
Then $Y=(Y_1, \dots, Y_n)$ are uniformly distributed on the sphere.
Using symmetric property we have $\E[Y_i^2] = \frac{1}{n}$.
Our problem is to compute the higher order moments of $\E[Y_i^k]$
where $k$ is even. Notice that $\E[Y_i^k]=0$ if $k$ is odd.

First we consider the simple case in which $n=2$ and we show that
\begin{equation}\label{eq:n2}
\int_{\R}\int_{\R} \frac{x^{2k}}{(x^2+y^2)^k}
    p(x)p(y)dxdy = \frac{(2k-1)!!}{(2k)!!}
\end{equation}
where $p(x) = \frac{1}{\sqrt{2\pi}} \exp(-\frac{x^2}{2})$.
We can let $y=xt$ and the integral is transformed to
\begin{align*}
\int_{\R}\int_{\R} \frac{x^{2k}}{(x^2+y^2)^k} p(x)p(y)dxdy & =
\frac{2}{\pi}\int_{0}^{+\infty}\int_{0}^{+\infty} \frac{x}{(1+t^2)^k}
\exp(-\frac{1}{2}x^2(1+t^2))dxdt \\
&=
\frac{2}{\pi}\int_{0}^{+\infty} \frac{1}{(1+t^2)^{k+1}}dt \\
&= \frac{2}{\pi}\int_{0}^{\frac{\pi}{2}} cos^{2k}(u)dt \\
\end{align*}
The integral of $sin(x)$ over $[0, \frac{\pi}{2}]$ is a well-known
\href{https://math.stackexchange.com/questions/50447/
integration-of-powers-of-the-sin-x}{formula}.
And we conclude that the right hand size of equation \eqref{eq:n2} equals
$\frac{(2k-1)!!}{(2k)!!}$.

Next we consider the case when $n>2$.
Let $Z = X_2^2 + \dots + X_{n-1}^2$ and $Z$ is chi-square distribution.
We have
\begin{equation}\label{eq:n3}
\int_{\R}\int_{0}^{+\infty} \frac{x^{2k}}{(x^2+z)^k} p(x)q(z)dxdz
= \frac{\Gamma(\frac{n}{2}) \Gamma(\frac{2k+1}{2})}
{\sqrt{\pi} \Gamma(\frac{2k+n}{2})} = \prod_{i=0}^{k-1} \frac{1+2i}{n+2i}
\end{equation}
where $q(z) = \frac{1}{2^{\frac{n-1}{2}}\Gamma(\frac{n-1}{2})}
z ^{\frac{n-1}{2} - 1} \exp(-\frac{z}{2})$
This time we let $z=x^2 t$ and the integral is transformed to
\begin{align*}
\int_{\R}\int_{0}^{+\infty} \frac{x^{2k}}{(x^2+z)^k} p(x)q(z)dxdz & = \\
\frac{1}{2^{\frac{n}{2}-1}\sqrt{\pi}\Gamma(\frac{n-1}{2})}
\int_{0}^{+\infty}t^{\frac{n-1}{2}-1}
(1+t)^{-k}dt\int_{0}^{+\infty}x^{n-1} \exp(-\frac{1}{2}x^2(1+t)) dx &=
\\
\frac{1}{2^{\frac{n}{2}-1}\sqrt{\pi}\Gamma(\frac{n-1}{2})}
\int_{0}^{+\infty}t^{\frac{n-1}{2}-1}(1+t)^{-k-\frac{n}{2}}dt
\int_{0}^{+\infty}x^{n-1} \exp(-\frac{1}{2}x^2) dx &=
\\
\frac{2}{\sqrt{\pi}\Gamma(\frac{n-1}{2})}
\int_{0}^{+\infty}t^{\frac{n-1}{2}-1}(1+t)^{-k-\frac{n}{2}}dt
\int_{0}^{+\infty}x^{n-1} \exp(-x^2) dx &=
\\
\frac{\Gamma(\frac{n}{2})}{\sqrt{\pi}\Gamma(\frac{n-1}{2})}
\int_{0}^{+\infty}t^{\frac{n-1}{2}-1}(1+t)^{-k-\frac{n}{2}}dt
&= \\ (t=\tan^2 x)
\frac{\Gamma(\frac{n}{2})}{\sqrt{\pi}\Gamma(\frac{n-1}{2})}
2 \int_0^{\frac{\pi}{2}} (\sin x)^{n-2} (\cos x)^{2k} dx
&= \\ \frac{\Gamma(\frac{n}{2})}{\sqrt{\pi}\Gamma(\frac{n-1}{2})}
B(\frac{n-1}{2}, \frac{2k+1}{2})
&= \\ \frac{\Gamma(\frac{n}{2})\Gamma(\frac{2k+1}{2})}
{\sqrt{\pi} \Gamma(\frac{2k+n}{2})}
\end{align*}

Actually, we can caculate the exact distribution of $Y_1$.
Suppose $w(n, q) = \frac{1}{2^{nq/2} \Gamma_q(\frac{n}{2})}$ which is
called Wishart constant. $\Gamma_q$ is the multivariate Gamma function.
The Proposition 7.3, Chapter 7 of \cite{eaton1989group} says:
\begin{proposition}\label{prop:dens}
Suppose $X$ is a random orthogonal matrix,
$Y$ is the $p\times q$ upper submatrix of $X$.
$q\leq p $ and $p+q \leq n$.
Then the density function of $Y$ is given by

\begin{equation}
f(y) = \frac{w(n-p, q)}{\sqrt{2\pi}^{pq}w(n,q)}
\abs{I_q - y^T y}^{\frac{n-p-q-1}{2}}I_0(y^T y)
\end{equation}
where $I_0$ is an indicator function
of the $q\times q $ symmetric matrices.
$I_0(A)=1$ if all eigenvalues of $A$ falls within $(0,1)$.
Otherwise $I_0(A)=0$.
\end{proposition}
Using this proposition we can get the density function of $Y_1$ as
\begin{equation}
p(y) = \frac{\Gamma(\frac{n}{2})}{\sqrt{\pi}\Gamma(\frac{n-1}{2})}
(1-y^2)^{\frac{n-3}{2}}
\end{equation}
Using the property of beta function
$\int_{-1}^{1} y^{2k} p(y)dy =
\frac{\Gamma(\frac{n}{2})\Gamma(\frac{2k+1}{2})}
{\sqrt{\pi} \Gamma(\frac{2k+n}{2})}$,
which is same with Equation \eqref{eq:n3}.
Notice that $\hat{Y}_1 = Y_1^2 (-1 \leq Y_1 \leq 1)$ is beta distribution
$B(\frac{1}{2}, \frac{n-1}{2})$.
The high order moment of $\hat{Y}_1$ is a well known result.
See \href{https://en.wikipedia.org/wiki/Beta_distribution#Higher_moments}
{Higher moments of Beta distribution}.
We also consider the case when the moments of
$\hat{Y} = Y_1^2 + \dots + Y_r^2$ is required.
Let $Z = X_{r+1}^2 + \dots + X_n^2$. We have
\begin{equation}\label{eq:n4}
\int_{0}^{+\infty}\int_{0}^{+\infty} \frac{x^t}{(x+z)^t} p(x)q(z) dxdz =
\frac{\Gamma(\frac{2t+r}{2})\Gamma(\frac{n}{2})}
{\Gamma(\frac{2t+n}{2})\Gamma(\frac{r}{2})} =
\prod_{i=0}^{t-1} \frac{2i+r}{2i+n}
\end{equation}
where $p(x)$ is pdf of chi square distribution with degree $r$ and $q(z)$
is pdf of chi square distribution with degree $n-r$.
\begin{align*}
\int_{0}^{+\infty}\int_{0}^{+\infty} \frac{x^t}{(x+z)^t} p(x)q(z) dxdz =
& \\ \frac{1}{2^{n/2} \Gamma(\frac{n-r}{2})\Gamma(\frac{r}{2})}
\int_{0}^{+\infty}\int_{0}^{+\infty}
\left(\frac{x}{x+z}\right)^t
x^{\frac{r}{2}-1}\exp(-\frac{x}{2})
z^{\frac{n-r}{2}-1}\exp(-\frac{z}{2})dxdz =
& \\ (z=xy) \frac{1}{2^{n/2}\Gamma(\frac{n-r}{2})\Gamma(\frac{r}{2})}
\int_{0}^{+\infty}\int_{0}^{+\infty}
\left(\frac{1}{1+y}\right)^t
x^{\frac{n}{2}-1}\exp(-\frac{x(1+y)}{2})y^{\frac{n-r}{2}-1}dxdy =
& \\  \frac{1}{2^{n/2} \Gamma(\frac{n-r}{2})\Gamma(\frac{r}{2})}
\int_{0}^{+\infty}\left(\frac{1}{1+y}\right)^t
y^{\frac{n-r}{2}-1} dy
\int_{0}^{+\infty} x^{\frac{n}{2}-1}\exp(-\frac{x(1+y)}{2})dx = & \\
\frac{\Gamma(\frac{n}{2})}{\Gamma(\frac{n-r}{2})\Gamma(\frac{r}{2})}
\int_{0}^{+\infty}\left(\frac{1}{1+y}\right)^{t+\frac{n}{2}}
y^{\frac{n-r}{2}-1} dy =
& \\  \frac{\Gamma(\frac{n}{2})}{\Gamma(\frac{n-r}{2})\Gamma(\frac{r}{2})}
B(\frac{n-r}{2}, \frac{2t+r}{2}) =
& \\ \frac{\Gamma(\frac{2t+r}{2})\Gamma(\frac{n}{2})}
{\Gamma(\frac{2t+n}{2})\Gamma(\frac{r}{2})}
\end{align*}

From Proposition \ref{prop:dens}, let $p=r, q=1$.
The joint distribution of $Y_1, \dots, Y_r$ can be written as:
$$
p(y_1, \dots, y_r) = \frac{\Gamma(\frac{n}{2})}{\sqrt{\pi}^r
\Gamma(\frac{n-r}{2})}(1- y_1^2 - \dots - y_r^2)^{\frac{n-r}{2} -1}
$$
To compute $\E[(Y_1^2 + \dots Y_r^2)^t]$, we
make the spherical transformation and
integrate over all those angle components:
\begin{align*}
\frac{\Gamma(\frac{n}{2})}{\sqrt{\pi}^r\Gamma(\frac{n-r}{2})}
\frac{r\pi^{r/2}}{\Gamma(\frac{r}{2}+1)}
\int_{0}^1 y^{2t+r-1}(1- y^2)^{\frac{n-r}{2} -1} dy =
\frac{\Gamma(\frac{2t+r}{2})\Gamma(\frac{n}{2})}
{\Gamma(\frac{2t+n}{2})\Gamma(\frac{r}{2})}
\end{align*}
The result is the same with Equation \eqref{eq:n4}.
We can further show that $\hat{Y}$
is beta distribution with $B(\frac{r}{2}, \frac{n-r}{2})$
using the techniques similar to
\href{https://en.wikipedia.org/wiki/
Proofs_related_to_chi-squared_distribution
#Derivation_of_the_pdf_for_k_degrees_of_freedom}
{Derivation of pdf for chi-square distributions}.
Then the higher order moment of $\hat{Y}$ follows naturally.
The distribution of $\hat{Y}$ can also be got
from Proposition 7.2 in \cite{eaton1989group} by setting $p=r, q=1$.


We like to compute $\E[X_1^{2k_1} X_2^{2k_2}]$.
Let $ Z = X_3^2 + \dots + X_n^2$ which is a chi square distribution
with $n-2$ degree of freedom.
Then we have
\begin{align}\label{eq:cor}
&\int_{x,y\in \mathbb{R}}\int_{z\geq 0}
\frac{x^{2k_1} y^{2k_2}}{(x^2+y^2+z)^{k_1 + k_2}} p(x)p(y)q(z) dxdydz \notag\\
&=\frac{n-2}{2\pi} \frac{\Gamma(k_1 + \frac{1}{2})
\Gamma(k_2 + \frac{1}{2})
\Gamma(\frac{n}{2}-1)}{\Gamma(k_1 + k_2 + \frac{n}{2})}
\end{align}
We let $z=(x^2+y^2)t$ first.
\begin{align*}
\int_{x,y\in \mathbb{R}}\int_{z\geq 0}
\frac{x^{2k_1} y^{2k_2}}{(x^2+y^2+z)^{k_1 + k_2}} p(x)p(y)q(z) dxdydz & =\\
\frac{1}{2^{\frac{n-2}{2}}\Gamma(\frac{n-2}{2})}\frac{1}{2\pi}
\int_{x,y\in \mathbb{R}}\int_{t\geq 0}
\frac{x^{2k_1} y^{2k_2}t^{-2+\frac{n}{2}}}
{(x^2+y^2)^{k_1 + k_2+1-\frac{n}{2}}(1+t)^{k_1 + k_2}}
\exp(-\frac{1}{2}(x^2+y^2)(1+t))dxdydt =
\end{align*}
Secondly we make the transformation $y^2=x^2u$ with $u>0$.
\begin{align*}
\frac{1}{2^{\frac{n-2}{2}}\Gamma(\frac{n-2}{2})}\frac{2}{2\pi}
\int_{x,u,t\geq 0}\frac{x^{n-1}
u^{k_2 - \frac{1}{2}}t^{-2+\frac{n}{2}}}
{(1+u)^{k_1 + k_2+1-\frac{n}{2}}(1+t)^{k_1 + k_2}}
\exp(-\frac{1}{2}x^2(1+u)(1+t))dxdudt =
\end{align*}
We can first integrate about $x$ first using Gamma function
(let $\sigma^2 = \frac{1}{(1+u)(1+t)}$)
and get
\begin{align*}
\frac{2^{n/2} \Gamma(\frac{n}{2})}{2^{\frac{n-2}{2}}
\Gamma(\frac{n-2}{2})}\frac{1}{2\pi}\int_{u,t\geq 0}
\frac{u^{k_2 - \frac{1}{2}}t^{-2+\frac{n}{2}}\sigma^n}
{(1+u)^{k_1 + k_2+1-\frac{n}{2}}(1+t)^{k_1 + k_2}}dudt = \\
\frac{n-2}{2\pi}\int_{u,t\geq 0}
\frac{u^{k_2 - \frac{1}{2}}t^{-2+\frac{n}{2}}}
{(1+u)^{k_1 + k_2+1}(1+t)^{k_1 + k_2 + \frac{n}{2}}}dudt = \\
\frac{n-2}{2\pi}
B(k_2 + \frac{1}{2}, k_1 + \frac{1}{2}) B(-1+\frac{n}{2}, k_1 + k_2 + 1) =\\
\frac{n-2}{2\pi} \frac{\Gamma(k_1 + \frac{1}{2})
\Gamma(k_2 + \frac{1}{2})\Gamma(\frac{n}{2}-1)}
{\Gamma(k_1 + k_2 + \frac{n}{2})}
\end{align*}
From Equation \eqref{eq:cor}, we choose $k_1 = k_2 = k$ and we have
$\E[X_1^{2k}X_2^{2k}]=\frac{((2k-1)!!)^2}{\prod_{i=0}^{2k-1} (n + 2i)}$

We consider the decomposition of a $2\times 2$ symmetric matrix:
$$
\begin{pmatrix}
x & y \\
y & z
\end{pmatrix} =
\begin{pmatrix}
\cos \theta & -\sin \theta \\
\sin \theta & \cos \theta
\end{pmatrix}
\begin{pmatrix}
\lambda_1 & 0 \\
0 & \lambda_2
\end{pmatrix}
\begin{pmatrix}
\cos \theta & \sin \theta \\
-\sin \theta & \cos \theta
\end{pmatrix}
$$
We then have
\begin{align*}
x & = \cos^2\theta \lambda_1 + \sin^2 \theta \lambda_2 \\
y & = \cos\theta \sin\theta (\lambda_1 - \lambda_2) \\
z & = \sin^2\theta \lambda_1 + \cos^2 \theta \lambda_2
\end{align*}
If we have an integral over all $2\times 2$ positive definite matrices,
we can convert the integral to $\theta, \lambda_1, \lambda_2$.
The determinant of the Jacobi matrix is $\abs{\lambda_1 - \lambda_2}$.
The domain for $(\theta, \lambda_1, \lambda_2)$ is
$\lambda_1 \geq \lambda_2 \geq 0$ and $\theta \in [0,\pi]$.
We show that $\Gamma_2(a) = \sqrt{\pi}\Gamma(a) \Gamma(a-\frac{1}{2})$ where
$\Gamma_2(a) = \int_{S>0} \exp(-\tr(S)) \abs{S}^{a-\frac{3}{2}} dS$.

First we transform the integral to $(\lambda_1, \lambda_2, \theta)$ space and
integrate out $\theta$:
$$
\Gamma_2(a) = \pi \int_{\lambda_1 \geq \lambda_2} (\lambda_1 -\lambda_2)
\exp(-\lambda_1 - \lambda_2)
(\lambda_1 \lambda_2)^{a-\frac{3}{2}} d\lambda_1 d\lambda_2
$$
Next we make another transformation:
$\lambda_1 = r \cos^2 \phi, \lambda_2 = r \sin^2 \phi$.
The determinant of Jacobi is $2r\sin\phi \cos\phi$ where
$ r >0, \phi \in [0, \frac{\pi}{4}]$.
Then we have
\begin{align*}
\Gamma_2(a) =&  2\pi \int_0^{\frac{\pi}{4}}
(\cos^2\phi - \sin^2 \phi) (\cos\phi\sin\phi)^{2a-2} d\phi
\int_0^{+\infty} r^{2a-1} \exp(-r) dr \\
= & 2\pi \Gamma(2a) \int_0^{\frac{pi}{4}}
\cos(2\phi) \frac{1}{2^{2a-2}}\sin^{2a-2} (2\phi)d\phi \\
= & \pi\frac{\Gamma(2a)}{2^{2a-2}} \int_0^1 x^{2a-2}dx \\
= & \frac{\pi\Gamma(2a-1)}{2^{2a-2}}
\end{align*}
Using Legendre duplication formula $\Gamma(z) \Gamma(z+\frac{1}{2}) =
2^{1-2z} \sqrt{\pi} \Gamma(2z)$, let $ z = a - \frac{1}{2} $ we have
$\Gamma_2(a) = \sqrt{\pi}\Gamma(a)\Gamma(a-\frac{1}{2})$.

For $p \times p$ symmetric matrix, we define the multivariate Beta function
as
\begin{equation}
B_p(a, b) = \int_{X \in S_{p,p}^{+} }\abs{I-X}^{a-\frac{p+1}{2}}
\abs{X}^{b-\frac{p+1}{2}}dX
\end{equation}
The integration is about $p \times p$ positive definite matrix.
We will show that
\begin{equation}
B_p(a,b) = \frac{\Gamma_p(a)\Gamma_p(b)}{\Gamma_p(a+b)}
\end{equation}
where $\Gamma_p(a)$ is defined as:
\begin{equation}
\Gamma_p(a) =  \int_{X \in S_{p,p}^{+} }\abs{X}^{a-\frac{p+1}{2}}
\exp(-tr(X))dX
\end{equation}
The proof is similar with
\href{https://en.wikipedia.org/wiki/Beta_function
#Relationship_between_gamma_function_and_beta_function}
{one dimensional case}.
First we consider
$\Gamma_p(a)\Gamma_p(b) =
\int_{X,Y \in S_{p,p}^{+} }\abs{X}^{a-\frac{p+1}{2}}
\abs{Y}^{b-\frac{p+1}{2}}\exp(-tr(X+Y))dXdY$
Then we make the transformation $X = L(I-T)L^T, Y=LTL^T$
where $Z=LL^T$. We actually transform  the integral to $(T,Z)$ space
where $I-T, T,Z$ are positive definite.
The Jacobi of this transformation is $|Z|^{\frac{p+1}{2}}$
from the Proposition 5.11, \cite{eaton}. Therefore, the integral transforms to
$$
\int_{Z,T, I-T \in S_{p,p}^{+} }\abs{I-T}^{a-\frac{p+1}{2}}
\abs{T}^{b-\frac{p+1}{2}}\abs{Z}^{a+b-\frac{p+1}{2}}\exp(-tr(Z))dZdT
$$ which equals $\Gamma_p(a+b)B(a,b)$ by definition.

The multivariate Beta function is used as
a distribution for sub-blocks of $A=X'X'^T$ where
$X'$ is $n \times k$ sub-matrix of random orthogonal matrix $X$
($n\geq k+2, k\geq 2$).
From Proposition 7.2 of \cite{eaton1989group}
we conclude that the distribution for $2\times 2$ upper matrix of $A$ is
\begin{equation}
p(A_{11}, A_{12}, A_{22}) =
C_0 I_0(A_{11}, A_{12}, A_{22})[A_{11}A_{22}-A_{12}^2]^{\frac{k-3}{2}}
[(1-A_{11})(1-A_{22})-A_{12}^2]^{\frac{n-k-3}{2}}
\end{equation}
where $C_0 = \frac{w(k, 2) w(n-k, 2)}{w(n, 2)}$
and $I_0$ is the indicator function.
The feasible region is described by:
$ 0 < A_{11} < 1, 0 < A_{22} < 1, A_{12}^2 < \min\{A_{11}A_{22},
1 - A_{11} - A_{22} + A_{11}A_{22}\}$

If only $A_{11}$(or $A_{22}$) is considered,
$A_{11}$ is Beta distribution with parameter $B(\frac{k}{2}, \frac{n-k}{2})$.
Then using the conclusion of higher order moment for Beta distribution,
we can get $\E[A_{11}^t] = \prod_{i=0}^{t-1}\frac{2t+k}{2t+n}$.
The conclusion is the same with Equation \eqref{eq:n4}.

For $q=2$,
we have $B_2(a+1,b)=\frac{a(a-\frac{1}{2})}{(a+b)(a+b-\frac{1}{2})} B_2(a,b)$.
This statement is similar with
$B(a+1,b) = \frac{a}{a+b}B(a,b)$.

To simplify our notitation,
let $X = A_{11}, Y = A_{22}, Z = A_{12}$ and
$A = XY - Z^2, B = 1 - X - Y + A$.
Let $\E_{n,k}[f(X,Y,Z)]$ denotes the expectation
when the distribution parameters are $n, k$
and $f(X,Y,Z)$ is a function of the three input variables.
Then we have $\E_{n,k}[Af(X,Y,Z)] =
\frac{k(k-1)}{n(n-1)}\E_{n+2,k+2}[f(X,Y,Z)]$
and $\E_{n,k}[Bf(X,Y,Z)] = \frac{(n-k)(n-k-1)}{n(n-1)}\E_{n+2, k}[f(X,Y,Z)]$.
We can omit the subscript $n,k$ when there is no ambiguity.
Using this property we show that
\begin{equation}\label{eq:XZm2}
\E_{n,k}[X^m Z^2] = \frac{n-k}{n-1}
\frac{\prod_{i=0}^m (k+2i)}{\prod_{i=0}^{m+1} (n+2i)}
\end{equation}

Since $Z^2 = XY - A = X(1-X+A-B)-A$,
using $\E_{n,k}[X^{m+1}] = \frac{k+2m}{n+2m}\E_{n,k}[X^m]$ we have
\begin{align*}
\E_{n,k}[X^m Z^2] & = \E_{n,k}[X^{m+1}-X^{m+2}+AX^{m+1}-BX^{m+1}-AX^m] \\
& = (1-\frac{k+2m+2}{n+2m+2})\E_{n,k}[X^{m+1}] +
\frac{k(k-1)}{n(n-1)}\E_{n+2,k+2}[X^{m+1} - X^m]\\
& - \frac{(n-k)(n-k-1)}{n(n-1)} \E_{n+2,k}[X^{m+1}] \\
& = \frac{n-k}{n+2m+2}\E_{n,k}[X^{m+1}] -
\frac{k(k-1)}{n(n-1)}\frac{n-k}{n+2m+2}\E_{n+2,k+2}[X^m] \\
& - \frac{(n-k)(n-k-1)}{n(n-1)} \E_{n+2,k}[X^{m+1}] \\
\end{align*}
Since $\E_{n,k}[X^{m+1}] = \frac{k}{n}\E_{n+2,k+2}[X^m]$,
we can further simplify the above expression as:
\begin{align*}
\E_{n,k}[X^m Z^2] &=
\frac{k(n-k)}{n(n+2m+2)}(1-\frac{k-1}{n-1})\E_{n+2,k+2}[X^m] -
\frac{(n-k)(n-k-1)}{n(n-1)} \E_{n+2,k}[X^{m+1}]\\
& = \frac{k(n-k)^2}{n(n-1)(n+2m+2)}\prod_{i=1}^m \frac{k+2i}{n+2i} -
\frac{(n-k)(n-k-1)}{n(n-1)}\prod_{i=0}^m \frac{k+2i}{n+2i+2} \\
& = \frac{n-k}{n-1}\frac{\prod_{i=0}^m (k+2i)}{\prod_{i=0}^{m+1} (n+2i)}
\end{align*}
Using Equation \eqref{eq:XZm2}, we will show that
\begin{equation}\label{eq:XZm4}
\E[X^m Z^4] = \frac{3(n-k)(n+2-k)}{(n-1)(n+1)}
\frac{\prod_{i=0}^{m+1} (k+2i)}{\prod_{i=0}^{m+3} (n+2i)}
\end{equation}
Equation \eqref{eq:XZm4} follows from
$\E[X^m Z^4 ] = \E[Z^2 X^m(X-X^2 +AX-BX-A)]$.

Further we show that
\begin{equation}\label{eq:ZXtm}
\E[Z^{2t} X^m] = (2t-1)!! \prod_{i=0}^{t-1}\frac{n-k+2i}{n-1+2i}
\frac{\prod_{i=0}^{t-1+m} (k+2i)}{\prod_{i=0}^{2t-1+m}(n+2i)}
\end{equation}
We use induction to show Equation \eqref{eq:ZXtm} is true.
Suppose Equation \eqref{eq:ZXtm} holds for $\E[Z^{2t-2}X^m]$, then
\begin{align*}
\E[Z^{2t}X^m] &= \E[Z^{2t-2}X^m(X-X^2+AX-BX-A)] \\
&= \E[Z^{2t-2}(X^{m+1} - X^{m+2})] +
\E[A Z^{2t-2} (X^{m+1} - X^m)] - \E[BZ^{2t-2} X^{m+1}] \\
&= \left(1-\frac{k+2(t+m)}{n+2(2t+m-1)}\right)\E[Z^{2t-2}X^{m+1}] +
\frac{k(k-1)}{n(n-1)}\E_{n+2,k+2}[Z^{2t-2}(X^{m+1} - X^m)] \\
&-\frac{(n-k)(n-k-1)}{n(n-1)}\E_{n+2,k}[Z^{2t-2}X^{m+1}]\\
&=\frac{n-k+2(t-1)}{n+2(2t+m-1)}\E[Z^{2t-2}X^{m+1}] -
\frac{k(k-1)}{n(n-1)}\left(1-\frac{k+2(t+m)}{n+2(2t-1+m)}\right)
\E[Z^{2t-2}X^m] \\
&-\frac{(n-k)(n-k-1)}{n(n-1)}\E_{n+2,k}[Z^{2t-2}X^{m+1}]\\
&=\frac{n-k+2(t-1)}{n+2(2t+m-1)}
\left(1-\frac{k-1}{n+1+2(t-2)}\right)\E[Z^{2t-2}X^{m+1}] \\
&-\frac{(n-k-1)(n-k+2(t-1))}{(n+2(t-1))(n+2(2t-1+m)}\E[Z^{2t-2}X^{m+1}] \\
&=\frac{(2t-1)(n-k+2(t-1))}{(n+2(t-1)) (n+2(2t-1+m))}\E[Z^{2t-2}X^{m+1}]
\end{align*}
Using Equation \eqref{eq:XZm2} for $t-1$
we can get the same form of expression for $t$.

\bibliographystyle{plain}
\bibliography{exportlist}

\end{document}
