\documentclass{article}
\usepackage{amssymb}
\usepackage{amsmath}
\usepackage{amsthm}
\usepackage{mathtools}
\usepackage{hyperref}
\DeclarePairedDelimiter\abs{\lvert}{\rvert}
\DeclarePairedDelimiter\norm{\lVert}{\rVert}
\DeclarePairedDelimiter\inner{\langle}{\rangle}
\def\P{\mathcal{P}}
\def\E{\mathbb{E}}
\DeclarePairedDelimiter\floor{\lfloor}{\rfloor}
\DeclarePairedDelimiter\ceil{\lceil}{\rceil}
\DeclareMathOperator*{\diag}{diag}
\DeclareMathOperator*{\argmin}{argmin}
\newtheorem{lemma}{Lemma}
\newtheorem{remark}{Remark}
\newtheorem{proposition}{Proposition}
\title{Non-linearity in random variable approximation}
\author{Feng Zhao}

\begin{document}
\maketitle
\section{Problem}
Let $Y, X_1, \dots, X_k$ be n-dimensional uniformly distributed random vector on unit sphere. $X=(X_1, \dots, X_k)$ is an $n\times k$ random matrix truncated from a $n\times n$ random orthogonal matrix. $Y$ is independent with $X$. Each sample $x_1, \dots, x_k$ from $X$ have the property that $x_i \cdot x_j = 0$. Suppose $\norm{Y}=1$. We would like to compute the following quantity:
\begin{equation}\label{eq:Eminw}
\E[\min_w \norm{Y - \sigma (X w ) }^2]
\end{equation}
$\sigma$ is generally a non-linear scalar function. Its application on a vector is element-wise.
\begin{lemma}\label{lem:uniform}
Suppose $X$ is an $n$ by $k$ random matrix. The samples $x_1, \dots, x_k$ from $X$ have the property that $x_i \cdot x_j = 0, \norm{x_i}=1$ where $x_i$ is the $i$-th column.  We also require $X_i$ is uniformly distributed on an unit sphere. Let $A=X X^T$, then we have
\begin{equation}
E[A_{ij}]= \begin{cases}
\frac{k}{n} & i = j\\
0 & i\neq j 
\end{cases}
\end{equation}
\end{lemma}
\begin{proof}
We can use a generator model to simulate the sampling of $X$. First we random select $x_1$ from uniform distributed random variable on an $n$ dimensional sphere. Then $x_2$ should be selected from the $n-1$ dimensional sphere orthogonal to $x_1$ and so on.
It is easier to show $\E[X_1X_1^T] = \frac{1}{n} I_n$ since we can write $X_1 = \frac{(a_1, a_2, \dots a_n) }{\sqrt{a_1^2+\dots + a_n^2}}$ where $a_1, \dots a_n$ are i.i.d Gaussian.

To show $\E[A]=\frac{k}{n}I_n$ we need to show respectively $\E[X_iX_i^T]=\frac{1}{n} I_n$. We first show this equality holds for $i=2$. From the generator model, $X_2$ depends on $X_1$. By the Law of total expectation we have $\E[X_2 X_2^T] = \E_{X_1}[\E[X_2 X_2^T |X_1 = x_1]]$. $X_2 | x_1$ is a random variable distributed on an $n-1$ sphere. We assume $b_1, \dots, b_{n-1}$ is an orthogonal unit basis for this $n-1$ dimensional space,
then $b_1, \dots, b_{n-1}, x_1$ are an orthogonal unit basis for the $n$ dimensional space. Therefore we have
$\sum_{i=1}^{n-1} b_i b_i^T = I_n -  x_1 x_1^T $. The RHS is fixed for given $x_1$, which is irrelevant with the choice of $b_1, \dots, b_{n-1}$. We can show that the inner expectation $\E[X_2 X_2^T |X_1] = \frac{1}{n-1}(I_n - x_1 x_1^T)$ since we can choose $X_2 = \frac{1}{\sqrt{a_1^2 + \dots + a_{n-1}^2}} \sum_{i=1}^{n-1} a_i b_i$ ($b_i$ is fixed vector while $a_i$ is scalar random vector). Then taking the outer expectation we have $\E[X_2 X_2^T] = \frac{1}{n-1} (I_n - \frac{1}{n} I_n) = \frac{1}{n} I_n$.

For $i>2$, the proof is similar as we have $$
\E[X_i X_i^T] = \E_{X_1, \dots, X_{i-1}} [\E[X_i X_i^T | x_1, \dots x_{i-1}]]
$$
 which equals $\frac{1}{n-i+1}(I_n - \frac{i-1}{n} I_n) = \frac{1}{n} I_n$.
\end{proof}
\begin{remark}
To compute higher order moments $\E[A_{ii}^t]$. We can approximate it by $r^t$ where $r=\frac{k}{n}$. We require that $k$ is large and  $ t << k$.
Indeed, $A_{ii} = \sum_{j=1}^k x_{ij}^2$
\begin{align*}
\E[A^t_{ii}] & = \E[(\sum_{j=1}^k X_{ij}^2)^t] \\
\approx & k^t \E[X^2_{ij_1}X^2_{ij_2}\dots X^2_{ij_t}]
\end{align*}
Using the conclusion from Lemma \ref{lem:uniform}, we can get for example $\E[X_{11}^2 X_{12}^2]
= \E_{X_1}[X_{11}^2 \E[ X^2_{12}| X_1]] = \frac{1}{n-1}\E[X_{11}^2 ( 1 - X_{11}^2 )]$ when $n$ is large,
$X_{11} \sim N(0, \frac{1}{n})$ and $\E[X_{11}^4]$ is one order smaller than $\E[X_{11}^2]$ and can be  omitted. Therefore we get the result that $\E[X_{11}^2 X_{12}^2] \sim \frac{1}{n^2}$
\end{remark}
\begin{remark}
Exact solution for $t=2$. The fourth order moment for each component of uniform distribution on the sphere is $\frac{3}{n(n+2)}$. Then we have
\begin{align*}
\E[(X_{11}^2 + \dots + X_{1k}^2)^2] &= k\frac{3}{n(n+2)} + k(k-1)\frac{1}{n-1} (\frac{1}{n} - \frac{3}{n(n+2)}) \\
&=\frac{k(k+2)}{n(n+2)}
\end{align*}
In general, we can treat $X_{11}, \dots, X_{1k}$ as k components of a uniform random vector on the sphere. And we have 
\begin{equation}
\E[A_{ii}^t] = \prod_{i=0}^{t-1}\frac{k+2i}{n+2i}
\end{equation}
\end{remark}
\begin{remark}
$\E[A_{12}^2] = \frac{k}{n(n+2)} \Rightarrow \E[A_{12}^{2t}] \leq \E[A_{12}^2]^t$
\end{remark}
\begin{lemma}\label{lem:x2y2}
Suppose $(X,Y)$ is two-dimensional Gaussian vector, has zero mean and covariance vector $\Sigma$, then 
\begin{align*}
\E[X^2 Y^2] &= \Sigma_{11}\Sigma_{22} + 2\Sigma_{12}^2 \\
\E[X^3 Y + Y^3 X ] &= 3 (\Sigma_{11} + \Sigma_{22}) \Sigma_{12} \\
\E[X^3 Y^3] & = 6 \Sigma_{12}^3 + 9 \Sigma_{12} \Sigma_{11} \Sigma_{22}\\
\E[X^4 Y^4] & = 24 \Sigma_{12}^4 + 72 \Sigma_{12}^2 \Sigma_{11} \Sigma_{22} + 9\Sigma_{11}^2 \Sigma_{22}^2
\end{align*}
For more formulas, please see Isserlis' theorem, or \href{https://en.wikipedia.org/wiki/Multivariate_normal_distribution#Higher_moments}{Higher moments} in wikipedia.
\end{lemma}
\begin{remark}
Suppose $ i + j $ is even, using Isserlis's theorem we have
\begin{equation}
\E[X^i Y^j] = \sum_{k=0, k+i \textrm{ is even}}^{\min\{i,j\}}\frac{i! j!}{k! (i-k)!!(j-k)!!} \Sigma_{12}^k \Sigma_{11}^{(i-k)/2}\Sigma_{22}^{(j-k)/2}
\end{equation}
\end{remark}
\begin{proof}
Let $\Sigma = L^T L $ and $\binom{x}{y} = L^T \binom{x'}{y'}$, then $\binom{x'}{y'}$ has identity covariance. By Cholesky decomposition we can make $L$ be upper trianglar matrix (
$L_{12}=0$) with $L_{11}^2 = \Sigma_{11}, L_{11}L_{12} = \Sigma_{12}, L_{22}^2 = \Sigma_{22} - \frac{\Sigma_{12}^2}{\Sigma_{11}}$. Then we have $x = L_{11} x', y = L_{12} x' + L_{22} y'$. Therefore
\begin{align*}
\E[X^2 Y^2] & = L_{11}^2 (L_{12}^2\E[X'^4]+ L^2_{22}\E[X'^2]\E[Y'^2]) \\
& = L_{11}^2(3L_{12}^2 + L^2_{22}) \\
& = \Sigma_{11}\Sigma_{22} + 2\Sigma_{12}^2
\end{align*}
\end{proof}
\begin{lemma}\label{lem:abcd}
Suppose $X_1, X_2, Y_1, Y_2$ are gaussian random variables and the covariance of their join distribution is block diagonal:
$(X_1, X_2, Y_1, Y_2) \sim N(0, \binom{\Sigma_1, 0}{0, \Sigma_2})$. Then the following equality holds:
\begin{equation}
\E[f(X_1X_2)g(Y_1Y_2)] = \E[f(X_1 X_2)] \E[g(Y_1 Y_2)]
\end{equation}
\end{lemma}
\begin{proof}
\begin{align*}
\E[f(X_1X_2)g(Y_1Y_2)] & = \int \frac{f(x_1 x_2) g(y_1 y_2) }{(2\pi)^2 \sqrt{\abs{\Sigma_1}\abs{\Sigma_2}}}\exp(-\frac{1}{2}(x_1, x_2) \Sigma_1^{-1} \binom{x_1}{x_2}-\frac{1}{2}(y_1, y_2) \Sigma_1^{-1} \binom{y_1}{y_2})dx_1 d x_2 dy_1 dy_2 \\
& = \int \frac{f(x_1 x_2)  }{(2\pi) \sqrt{\abs{\Sigma_1}}}\exp(-\frac{1}{2}(x_1, x_2) \Sigma_1^{-1} \binom{x_1}{x_2})dx_1 d x_2  \\
&\cdot  \int \frac{ g(y_1 y_2)  }{(2\pi) \sqrt{\abs{\Sigma_1}}}\exp(-\frac{1}{2}(y_1, y_2) \Sigma_1^{-1} \binom{y_1}{y_2})dy_1 d y_2 \\
& = \E[f(X_1 X_2)] \E[g(Y_1 Y_2)]
\end{align*}
We can also use the property of Gaussian distribution directly: $X=(X_1, X_2)$ and $Y=(Y_1, Y_2)$ are both joint Gaussian random vector and these two are independent with each other from their join distribution. Then they are uncorrelated.
\end{proof}

We assume $\sigma(z) = z + \epsilon \xi(z)$. When $\epsilon = 0$, the optimal $w$ for given $X, Y$ is 
$\bar{w} =X^T Y $. For small $\epsilon$, we assume $ w = \bar{w} + \epsilon \hat{w} + \epsilon^2 \tilde{w}$. Then we can expand $\norm{Y - \sigma (X w ) }^2$ as follows:
\begin{align*}
\norm{Y - \sigma (X w ) }^2 & = \norm{Y - X \bar{w} - \epsilon X \hat{w} - \epsilon^2 X \tilde{w} - \epsilon \xi(X \bar{w} + \epsilon X\hat{w}) }^2 \\
& = \norm{ Y - X  \bar{w}  - \epsilon (X\hat{w} + \xi(X \bar{w} )) - \epsilon^2(X\tilde{w} + \nabla \xi(X \bar{w} )X\hat{w})}^2 \\
& = \norm{Y-X \bar{w}  }^2 - 2 \epsilon (X \hat{w} + \xi(X \bar{w}))^T (Y-X \bar{w}) +\\
& +\epsilon^2(\norm{X\hat{w} + \xi(X \bar{w})}^2 - 2 (X \tilde{w}+ \nabla \xi(X \bar{w})X\hat{w})^T(Y-X \bar{w}))
\end{align*}
In the above formula,  we expand $\xi$ at $X \bar{w}$. The Jacobi is actually a diagonal matrix whose $i$-th entry is $\xi'([X\bar{w}]_i)$. $[\cdot]_i$ represents the $i$-th element of a vector.
We first notice that for given $X$, $X\bar{w}$ is the projection of $Y$ onto linear subspace spanned by columns of $\bar{X} =X$. We use $\tilde{Y}$ to denote the mirror of $Y$ about this linear subspace. Then we have
$\bar{X}\bar{X}^T Y = \bar{X}\bar{X}^T \tilde{Y}$ and $(Y- \bar{X}\bar{X}^TY) = -(\tilde{Y} - \bar{X}\bar{X}^T \tilde{Y})$. This refers to $\xi(X\bar{w})^T (Y-X\bar{w})$. For $(X\hat{w})^T (Y-X\bar{w})$, we use the property that $  \bar{X}^T \bar{X} = I_k$.
By symmetric property, the integration of coefficient of $\epsilon$ with respect to $Y$ is zero (given $X$). 
 For the coefficient of $\epsilon^2$ we have
\begin{equation*}
\textrm{Coeff}(\epsilon^2)  =  \E[\norm{X\hat{w} + \xi(\bar{X}\bar{X}^TY)}^2 - 2 (\nabla \xi(\bar{X}\bar{X}^TY)X\hat{w})^T(Y-\bar{X}\bar{X}^TY)]
\end{equation*}
We have written Equation \eqref{eq:Eminw} in $$ \norm{Y - X \bar{w}}^2 + \textrm{Coeff}(\epsilon^2) \epsilon^2 + o(\epsilon^2)$$.

Minimizing $\E[\textrm{Coeff}(\epsilon^2)]$ we have
\begin{equation}
\min \textrm{Coeff}(\epsilon^2) = \E[\norm{\xi(\bar{X}\bar{X}^TY)}^2 - \norm{\bar{X}^T\xi(\bar{X}\bar{X}^TY)}^2 - \norm{\bar{X}^T \nabla\xi(\bar{X}\bar{X}^TY)(Y-\bar{X}\bar{X}^TY)}^2]
\end{equation}
The minimal value is achieved at 
$$
\hat{w} =  X^T(\nabla\xi(\bar{X}\bar{X}^T Y)
(Y-\bar{X}\bar{X}^T Y) - \xi(\bar{X}\bar{X}^T Y))
$$
Below we assume $\bar{X}$ is given (the expectation is taken first about $Y$ and then about $\bar{X}$), the columns of $\bar{X}$ are orthogonal by the given condition ($\bar{X}^T\bar{X}=I_k$) and $Y_1, \dots, Y_n $ be component variable of $Y$. The covariance matrix of $Y$ is $I_n$ assuming $Y_i$ are i.i.d. std Gaussian. We use $I_1, I_2, I_3$ to denote the three terms of $\textrm{Coeff}(\epsilon^2)$ with given $X$ and $A=\bar{X}\bar{X}^T$ is an $n\times n$ matrix. That is $ \textrm{Coeff}(\epsilon^2) = \E_{X} [I_1 + I_2 + I_3]$.
Let $Z = AY$. Then the covariance matrix of $Z$ is $A\E[YY^T]A^T = \frac{1}{n}AA^T = \frac{1}{n}A$ (from Lemma \ref{lem:uniform}).
Also, both $Y$ and $Z$ have zero mean (vector). We then have
\begin{equation*}
I_1 =\E[\norm{\xi(Z)}^2] = \sum_{i=1}^n \E[\xi^2(z_i)]
\end{equation*}
where $Z_i$ has the distribution $p_i$.
 
For $I_2$
we have
\begin{equation*}
-I_2   = \sum_{i,j=1, i \neq j}^n A_{ij}\E[\xi(z_i)\xi(z_j)] + \sum_{i=1}^n A_{ii}  \E[\xi^2(z_i)]
\end{equation*}
where $Z_i, Z_j$ have the joint distribution $p_{ij}$.

%$A_{ii} = \sum_{j=1}^k x^2_{ij} < 1$. 
Then we have
\begin{equation}\label{eq:I1plusI2}
I_1+ I_2 =   \sum_{i=1}^n (1-A_{ii}) \E[\xi^2(z_i)]  - \sum_{i,j=1, i \neq j}^n A_{ij}\E[\xi(z_i)\xi(z_j)] 
\end{equation}
For $I_3$, we have
\begin{align*}
-I_3 & = \E_Y[\norm{X^T \nabla\xi(XX^TY)(Y-XX^TY)}^2] \\ 
 & = \sum_{i,j=1, i \neq j}^n A_{ij}\Sigma_{ij} + \sum_{i=1}^n A_{ii}\Sigma_{ii} 
\end{align*}
For $\Sigma_{ii}$ we have
\begin{align}\label{eq:sigmaii}
\Sigma_{ii} &=  \E[ [(I-A)Y]_i^2 [\nabla \xi(AY)]_{i,i}^2] \\
&= \frac{1}{n}(1-A_{ii}) \E[ [\nabla \xi(AY)]_{i,i}^2]
\end{align}

For $\Sigma_{ij}$ with $i\neq j$ we have
\begin{align*}
\Sigma_{ij} &=  \E[ [(I-A)Y]_i [(I-A)Y]_j [\nabla \xi(AY)]_{i,i} [\nabla \xi(AY)]_{j,j}]  \\
&= - \frac{1}{n}A_{ij} \E[[\nabla \xi(AY)]_{i,i} [\nabla \xi(AY)]_{j,j}]
\end{align*}

We then have
\begin{align*}
I_1+I_2+I_3 &= \sum_{i=1}^n (1-A_{ii})(\E[\xi^2(z_i)] -\frac{1}{n} A_{ii} \E[ [\nabla \xi(AY)]_{i,i}^2]) \\
&- \sum_{i,j=1, i \neq j}^n A_{ij} (\E[\xi(z_i)\xi(z_j)] - \frac{1}{n}A_{ij}\E[[\nabla \xi(AY)]_{i,i} [\nabla \xi(AY)]_{j,j}])
\end{align*}
By symmetric property, we can simplify the above equation as:
\begin{align}\label{formularI123}
I_1+I_2+I_3 &=n(1-A_{11})(\E[\xi^2(z_1)] - \frac{1}{n} A_{11} \E[ [\nabla \xi(AY)]_{1,1}^2]) \\
&- n(n-1)A_{12}(\E[\xi(z_1)\xi(z_2)] -  \frac{1}{n}A_{12}\E[[\nabla \xi(AY)]_{1,1} [\nabla \xi(AY)]_{2,2}])\nonumber
\end{align}

We consider $\xi(z) = a_0 + a_1 z + a_2 z^2 + \dots + a_{m-1} z^{m-1} + a_m z^m $, there will be an $(m+1) \times (m+1) $ matrix $M$, a vector  $q = [a_0, a_1, \dots, a_m]^T$ and  we can write $\textrm{Coeff}(\epsilon^2) = I_1 + I_2 + I_3$ as the quadratic form $ q^T M q $. The element $M_{ij}$ is the coefficient of $a_ia_j$. Since $M$ is symmetric we only consider $i\leq j$.

Using Equation \ref{formularI123}, we have 
\begin{align*}
M_{ij} &= n(1-A_{11}) (\E[Z_1^{i+j}] - ij \frac{1}{n}A_{11} \E[Z_1^{i+j-2}])  \\
&-n(n-1)A_{12}(\E[Z_1^i Z_2^j] - ij \frac{1}{n}A_{12}\E[Z_1^{i-1}Z_2^{j-1}])
\end{align*}
We can approximate $\E[Z_1^{2t}]$ where $Z=AY$ assuming $Y$ is a Gaussian vector with variance $\frac{1}{n}$. Then $\E[Z_1^{2t}] = A_{11}^k (2t-1)!!$.

Let $i+j=2t$. Then we have 
\begin{align*}
M_{ij} &= n(1-A_{11})\frac{A^t_{11}}{n^t}  ((2t-1)!!- ij (2t-3)!!)  \\
&+n(n-1)\frac{1}{n^t}A_{12} \sum_{s=0, s+i \textrm{ is even}}^{\min\{i,j\}} \frac{(s-1)i!j!}{s!(i-s)!!(j-s)!!} A_{12}^s A_{11}^{(i - s) /2}A_{22}^{(j - s)/2}
\end{align*}

For $M_{0j}$ where $j \neq 0$, it equals to zero if $j$ is an odd number, let $j=2t$ we can compute that $M_{0,2t}=\frac{1}{n^{t-1}} (1-r)r^t (2t-1)!! $. 

For $i \neq 0$, if $i+j$ is odd then $M_{ij} = 0$. Consider $i \neq j$ and $i+j = 2t$ we then have $M_{ij} = \frac{1}{n^{t-1}} (2t-1-ij) (1-r)r^t (2t-3)!! = -n(i-1)(j-1) (1-r)r^{(i+j)/ 2 } (i+j-3)!!$. Finally, consider $M_{ii}$ with $i \neq 1$. we have $M_{ii} = 
-\frac{1}{n^{i-1}} (i-1)^2 (1-r)r^i (2i-3)!!$.

Average over $A_{11}, A^2_{12}(\approx r' = \frac{k}{n^2}), A_{22}(\approx r = \frac{k}{n}) $
The summation must start with $t=3$. The order of $A_{12}^4$ in
The second term of $M_{ij}$ must start with $4$. Therefore the order of $n$ in second term is smaller than the first term and we can neglect the second term if $n$ is sufficiently large.

We summarize our deduction as follows:
\begin{equation}
M_{ij} = \begin{cases} 0 & i=1 \textrm{ or } j=1 \textrm{ or } i + j \textrm{ is odd} \\
 -\frac{1}{n^{(i+j)/2-1}}(i-1)(j-1) (1-r)r^{(i+j)/ 2 } (i+j-3)!! & i+j \textrm{ is even and } i,j \neq 1 \\
(1-r)n & i=j=0
\end{cases}
\end{equation}
For example, consider $m = 4$ we can write the $ 5 \times 5 $ matrix as follows ($r'=\frac{r}{n}$):
$$
(1-r)n\begin{pmatrix}
1 & 0 & r'  & 0 & 3r'^2\\
0 & 0 & 0  & 0 & 0\\
r' &  0 & - r'^2 & 0 & -9 r'^3 \\
0 & 0 & 0 & -12r'^3 & 0 \\
3r'^2 & 0 & -9 r'^3 & 0 & -135r'^4 
\end{pmatrix}
$$

Now we consider the normalization condition: $I_1 = 1$. Let $N$ be an $(m+1) \times (m+1)$ matrix.
Then we have $q^T N q = 1$ where $N_{ij} =\frac{1}{n^{(i+j)/2-1}}r^{(i+j) / 2} (i+j -1)!!$ when $i+j$ is even and $N_{ij} = 0$ when $i+j$ is odd. The special case is $N_{00} = 1$. Now if we rescale the parameter: $b_0 =  \sqrt{n}a_0, b_1 = \sqrt{r} a_1, b_2 = r \frac{1}{\sqrt{n}}a_2$ and $b_m = r^{m/2} \frac{1}{n^{(m-1)/2}}a_m$. Let $p = [b_0, b_1, \dots, b_m]$. Then $p^T \widetilde{N} p =1$ where $\widetilde{N}_{ij} = \delta(i,j) (i+j -1)!!$. Also, the object function $q^T M q$ is simplified as $q^T \widetilde{M} q$ where $\widetilde{M}_{ij} =  -(i-1)(j-1)(1-r) (i+j-3)!! $. This transformation makes the matrix irrelevant with $n$.

First we let $U^T U = \widetilde{N}$. We show that $U_{ij} = \delta(i,j)\frac{j!}{(j-i)!!\sqrt{i!}}$ where $U$ is an upper triangular matrix. $\delta(i,j)$ is an indicator function:
\begin{equation}
\delta_{ij} = \frac{1}{2}(1+(-1)^{i+j})=\begin{cases}
1 & i+j \textrm{ is even} \\
0 & i+j \textrm{ is odd}
\end{cases}
\end{equation}
We can show that $\widetilde{M} = (1-r)U^T \Lambda U$ where $\Lambda = \diag\{1,0, -1, -2, \dots, -(m-1)\}$.

Since $p^T U^T U p = 1$, and the minimal value $-(m-1)$ for $\textrm{Coeff}(\epsilon^2) = p^T U^T \Lambda U p $ is achieved when $U p = (0, \dots, 0, 1)^T = e_m$. Solving $ U p = e_m $ can we get the value of $p$. That is: $p_i = (-1)^{(m-i)/2} \frac{m!}{i!(m-i)!! \sqrt{m!}} $ when $i+m$ is even; $p_i = 0 $ when $i+m$ is odd. Transforming back to $a_i$ we have:
\begin{equation}
a_i =(-1)^{(m-i)/2} \delta(i,m) \frac{n^{(i-1)/2}}{r^{i/2}} \frac{m!}{i!(m-i)!! \sqrt{m!}}
\end{equation}

Suppose  $ m = 4$ we have $ \xi(z) = \frac{1}{\sqrt{24}}(\frac{3}{\sqrt{n}} - \frac{6\sqrt{n}}{r} z^2 + \frac{n^{3/2}}{r^2}z^4)  = \frac{1}{\sqrt{24}}((\frac{z^2}{r} - 3)^2 - 6)$. In such case we have $\E[\min \norm{Y - \sigma(Xw)}^2]=(1-r)(1-3\epsilon^2)$
\section{Orthogonal Polynomial}
Let $p_m(x) = \xi(\frac{z}{\sqrt{r}})$ where the highest order term of $p_m(x)$ is $m$. We can show that $p_m(x)$ is a series of orthogonal polynomial sequence with respect to the inner product 
$<x^i, x^j> = \delta(i,j)(i+j-1)!!$. Indeed, if $m_1 + m_2$ is odd, $<p_{m_1}(x), p_{m_2}(x)> = 0$.
We only need to consider $m_1 + m_2$ is even. WLOG, $m_1 < m_2$, we can pad the higher order terms of polynomial $p_{m_1}(x)$. Let $p_{m_i}$ be the resulting coefficient of $p_{m_i}(x)$
and $<p_{m_1}(x), p_{m_2}(x)> = p_{m_1}U^TUp_{m_2} = e^T_{m_1} e_{m_2} = 0$.

We can further show that $p_m(x)$ belongs to Hermite polynomials under some scaling.

\section{Experiment Results}
We consider $r=\frac{2}{3}, k=40, n=60, \epsilon=0.1$ and $m=2$. The sampling time is 6000.
For linear approximation the results is $0.3338$ ,  which corresponds to theoretical value $\frac{1}{3}$. 
If we add $\epsilon \xi_2(z)$ perturbation $\xi(z) = \frac{1}{\sqrt{2n}}(\frac{n}{r} z^2 -1)$, the improvement is $0.01 * \frac{1}{3}$. The simulation shows that the mse is $0.3317$. If we use sigmoid function instead of second order of Hermite polynomial, the result is $0.3923$, which is worse than linear case.

\section{Further discussion when $t$ is large}
Above we use some approximation $n, k >> 1$ and $ t << k$. 
We can weaken the assumptions. That is, we only preserve $ n >> 1$.
We can compute $N_{ij} =\frac{(2t-1)!!}{n^{t-1}} \prod_{s=0}^{t-1} \frac{(2s+k)}{(2s+n)}$ when $i+j = 2t$; otherwise $N_{ij}=0$. $M_{ij}=-\frac{1-r}{n^{t-1}} \prod_{s=0}^{t-1} \frac{(2s+k)}{(2s+n+2)} (i-1)(j-1) (i+j-3)!!$
\section{External Link}
Source document is available at 
\url{https://gitee.com/freewind201301/non-linear-activation-function/blob/master/non_linear.tex}

\appendix
\section{Decomposition of $\widetilde{N}$}
This section provides math induction proof for $\widetilde{N}= U^T U$ where 
\begin{align}
\widetilde{N}_{ij} & = \delta(i, j) (i+j-1)!! \\
U_{ij} & = \delta(i, j) \frac{j!}{(j-i)!!\sqrt{i!}} \textrm{ for } i\leq j 
\end{align}
\begin{proposition}
\begin{equation}
\delta(i, j) (i+j-1)!! = \sum_{k=0}^{\min\{i, j\}} \delta(k, i) \delta(k, j) \frac{i!}{(i-k)!!\sqrt{k!}}   \frac{j!}{(j-k)!!\sqrt{k!}}
\end{equation}
\end{proposition}
\end{document}
